%\documentclass[numberedappendix]{emulateapj}
\documentclass[letterpaper,12pt,preprint]{aastex}

\input{../paper-helpers/preamble.tex}

\begin{document}

\title{}
\author{Adrian M. Price-Whelan\altaffilmark{\colum,\adrn}}

% Affiliations
\newcommand{\colum}{1}
\newcommand{\adrn}{2}

\altaffiltext{\colum}{Department of Astronomy,
                      Columbia University,
                      550 W 120th St.,
                      New York, NY 10027, USA}
\altaffiltext{\adrn}{To whom correspondence should be addressed:
adrn@astro.columbia.edu}

\begin{abstract}
% Context
A measurement of the shape and radial profile of dark matter around the Milky
Way would provide a key test of cosmological models on galaxy-scales. Stellar
streams have been shown to precisely constrain these properties using simulated
data sets in a variety of analytic potential models.
% Aims
Here we show that modeling streams in these simplified potential models is unrealistically restrictive and makes these methods appear

this places un[...] constraints on the [...] and leads to
biased inferences and [...] precise measurements that do not [represent the true
precision...]
% Methods
We fit orbits to the leading tail of stellar streams simulated in a spherical
Plummer potential in [...].

% Results
We find that by allowing freedom [...]. "false sense of precision"

% Conclusions
\end{abstract}

\keywords{
  Galaxy: halo
  ---
  globular clusters: general
  ---
  stars: kinematics and dynamics
  ---
  Galaxy: structure
  ---
  Galaxy: kinematics and dynamics
}

\section{Introduction}\label{sec:introduction}

Dark matter halos simulated in the $\Lambda$CDM paradigm have density profiles and tend to be triaxial in
shape. When modeling the Milky Way halo, we use a handful of analytic potentials
based on simulation results. But this is incredibly restrictive: we don't know
that the radial profile of dark matter around the Milky Way is NFW! Break is
seen in stellar halo, slope may change in dark matter too. Outside of $\approx$50 kpc, probably lumpy -- what does spherically-averaged profile even mean?

Treating model as prior knowledge

For fixed potential form, number of orbits consistent with sky position alone is
very constraining! Super-informative prior

\section{Methods}\label{sec:methods}

\subsection{Stream simulations}\label{sec:stream-sims}

\subsection{Potential models}\label{sec:potentials}

\subsection{Orbit fitting}\label{sec:orbitfit}

\section{Results}\label{sec:results}

\subsection{}

\section{Conclusions}\label{sec:conclusions}

\acknowledgements
APW is supported by a National Science Foundation Graduate Research Fellowship
under Grant No.\ 11-44155. This work was supported in part by the National
Science Foundation under Grant No. PHYS-1066293. This research made use of
Astropy, a community-developed core Python package for Astronomy
\citep{astropy13}. This work additionally relied on Columbia University's
\emph{Hotfoot} and \emph{Yeti} compute clusters, and we acknowledge the Columbia
HPC support staff for assistance. \\

%\bibliographystyle{apj}
%\bibliography{refs}

\end{document}
